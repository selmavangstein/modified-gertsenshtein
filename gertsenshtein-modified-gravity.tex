%++++++++++++++++++++++++++++++++++++++++
% Don't modify this section unless you know what you're doing!
\documentclass[letterpaper,12pt]{article}
%\usepackage{dutchcal}
\usepackage{boondox-cal}
\usepackage{tabularx} % extra features for tabular environment
\usepackage{amsmath}  % improve math presentation
\usepackage{graphicx} % takes care of graphic including machinery
\usepackage[margin=1in,letterpaper]{geometry} % decreases margins
\usepackage{cite} % takes care of citations
\usepackage[final]{hyperref} % adds hyper links inside the generated pdf file
\hypersetup{
	colorlinks=true,       % false: boxed links; true: colored links
	linkcolor=blue,        % color of internal links
	citecolor=blue,        % color of links to bibliography
	filecolor=magenta,     % color of file links
	urlcolor=blue         
}
%++++++++++++++++++++++++++++++++++++++++


\begin{document}

\title{Some Title}
\author{A. Uthor, C. Ontributor}
\date{\today}
\maketitle

\begin{abstract}
Doing cool stuff to find cooler stuff
\end{abstract}


\section{Introduction}

The very important physical effect has applications to something in the world. 


\section{Background}

Gravitational theories and the Gertsenshtein effect combined in groundbreaking ways.
We use a (1,-1,-1,-1) metric.

The paper used the Einstein-Maxwell Lagrangian
\begin{equation}
\mathcal{L}=\sqrt{g}(R+\kappa F_{\mu \nu}F^{\mu \nu}),
\end{equation}
where $g$ is the determinant of the metric, $R$ is the Ricci Scalar, and $F$ is the electromagnetic field tensor. 

Linearizing the action and varying it gives the Maxwell field equations and the linearized gravitational wave eqaution. Using a transverse traceless Lorentz gauge for $h$, a small metric perturbation, with $\mathcal{h}_{\mu \nu}=\mathcal{h}_{\mu \nu}(t,z)$, they evaluated the components of the field equations (making a lot of assumptions about the EM wave and background), giving two mixing equations:
\begin{equation}
\ddot{ \mathcal{b_y}} - \mathcal{b_y}^{''}=B_0\mathcal{h}^{''}_{12}
\end{equation}
\begin{equation}
\ddot{ \mathcal{h}}_{12} - \mathcal{h}^{''}_{12}=4B_0 \mathcal{b_y}
\end{equation}

Here, we have a background magnetic field $\mathbf{B_0}=\mathbf{B}_{0x}$, and a magnetic perturbation $\mathbf{b}=\mathbf{b}_y (t,z)$. We have no electric field in the background, only in the wave.

We recreated the calculation with torsion, and got back the same equations.
\section{Our stuff}

We see what we get for the mixing equations with more complicated couplings in the lagrangian, with a torsion theory. We have a perturbation on the torsion given by 
\begin{equation}
\mathcal{T}^{\mu}_{\nu \sigma}=\epsilon^{\mu}_{\nu \sigma \lambda} \mathcal{Q}^{\lambda}+\delta^{\mu}_{[\nu} \mathcal{U}_{\sigma]}
\end{equation}
with
\begin{equation}
\mathcal{Q}_{\mu}=[\mathcal{q}_0(t,z),0,\mathcal{q}_2(t,z),0]
\end{equation}
and
\begin{equation}
\mathcal{U}_{\mu}=[\mathcal{u}_0(t,z),0,\mathcal{u}_2(t,z),0].
\end{equation}

When we add inn a background torsion $T$, we use $U=0$ and 
\begin{equation}
Q_{\mu}=[q_0,0,0,0]
\end{equation}


\textbf{Lagrangian 1 - $R_{\mu \nu} F^{\mu \nu}$}
\begin{equation}
\mathcal{L}=\sqrt{g}(R+\zeta R_{\mu \nu} F^{\mu \nu}+\kappa F{\mu \nu}F^{\mu \nu})
\end{equation}

\underline{With no background torsion:}

Maxwell and Einstein gives us nothing new on the face of it. The torsion field equations give us a bunch of mixings (notably with only first-order derivatives):

\begin{equation}
\frac{1}{2}  B_0 \mathcal{h}_{+}^{'} -  \partial_{t}\mathcal{E}_y=0
\end{equation}

\begin{equation}
\frac{1}{2} B_0 \mathcal{u}_0 + \frac{1}{2} B_0 \dot{\mathcal{h}}_{+}-\partial_{y} \mathcal{E}_z=0
\end{equation}

\begin{equation}
\frac{1}{2} B_0 \dot{\mathcal{h}}_{\times} - \dot{\mathcal{b}}=0
\end{equation}

We also have these three equations, which have some interesting impact on the original mixing equations we get from Maxwell and Einstein:
\begin{equation}
\frac{1}{2} B_0 \mathcal{h}_{\times}^{'}-\mathcal{b}^{'}+\partial_t \mathcal{E}_x=0
\end{equation}

\begin{equation}
\frac{1}{2} B_0 \mathcal{h}_{\times}^{'}-\mathcal{b}^{'}=0
\end{equation}

\begin{equation}
\frac{1}{2} B_0 \mathcal{h}_{\times}^{'}+\partial_t \mathcal{E}_x=0
\end{equation}

meaning that each of the three terms individually are zero, removing the z-derivatives of both $\mathcal{h}_{\times}$ and $\mathcal{b}$ in our original mixing equations, leaving

\begin{equation}
\ddot{ \mathcal{b_y}}=0
\end{equation}
\begin{equation}
\ddot{ \mathcal{h}}_{12} =4B_0 \mathcal{b_y}
\end{equation}


This lagrangian does not allow for a finite background torsion.



\textbf{Lagrangian 2 - $R_{\mu \nu} R^{\mu \nu}$}
\begin{equation}
\mathcal{L}=\sqrt{g}(R+\zeta R_{\mu \nu} R^{\mu \nu}+\kappa F{\mu \nu}F^{\mu \nu})=0
\end{equation}

\underline{With no background torsion:}

The torsion field equations have no EM-components, and only serve to help us simplify the other equations.
The maxwell field equations are unchanged. 

For the Einstein equation we get some higher-order derivatives, but no torsion contribution:

\begin{equation}
-4 \kappa B_0 \mathcal{b}-\mathcal{h}_{\times}^{''}+\ddot{\mathcal{h}}_{\times}+\zeta \mathcal{h}^{''''}+2  \zeta \ddot{\mathcal{h}}_{\times}^{''}+\zeta \ddddot{\mathcal{h}}_{\times}=0
\end{equation}


\underline{With background torsion:}

Computer still running. Only expecting change in Einstein one.


We also did a run with the indices on the second Ricci tensor swapped, but achieved the same mixing equations (different torsion field equations, but that had no effect on the mixing equations).


\textbf{Lagrangian 3} - $ R_{\mu \nu \sigma \lambda} R^{\mu \nu \sigma \lambda}$
\begin{equation}
\mathcal{L}=\sqrt{g}(R+\zeta R_{\mu \nu \sigma \lambda} R^{\mu \nu \sigma \lambda}+\kappa F{\mu \nu}F^{\mu \nu})
\end{equation}

\underline{With no background torsion:}

Maxwell still gives the same equations as expected.

Torsion still only gives some substitutions

Einstein is not done evaluating.



\underline{With background torsion:}
Not started


\textbf{Lagrangian 4} - $\epsilon_{\mu \nu \sigma \lambda} R^{\mu \nu} F^{\sigma \lambda}$
\begin{equation}
\mathcal{L}=\sqrt{g}(R+\zeta \epsilon_{\mu \nu \sigma \lambda} R^{\mu \nu} F^{\sigma \lambda}+\kappa F{\mu \nu}F^{\mu \nu})
\end{equation}

\underline{With no background torsion:}

No change in Maxwell, Einstein gave same mixing equation. Torsion field eqs gave some equations though:
\begin{equation}
2 \mathcal{b}
\end{equation}

FIX NOTATION - DERIVATIVES AND by VS b AND PROBABLY MORE

\underline{With background torsion:}
Not started. Is it worth trying?


\section{Conclusions}
The world is forever changed

%++++++++++++++++++++++++++++++++++++++++
% References section will be created automatically 
% with inclusion of "thebibliography" environment
% as it shown below. See text starting with line
% \begin{thebibliography}{99}
% Note: with this approach it is YOUR responsibility to put them in order
% of appearance.

\begin{thebibliography}{99}

\bibitem{bib} \emph{A source},  available at
\texttt{https://wevbarker.com/}.

\end{thebibliography}


\end{document}