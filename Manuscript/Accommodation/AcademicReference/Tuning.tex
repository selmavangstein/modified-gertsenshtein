Nicky was one of the eight Master's students whom I selected in the fall of 2022. During the academic year of 2022-23, Nicky and I worked closely together, with weekly meetings. Since graduating from his Master's degree, Nicky and I have continued to meet on a weekly or bi-weekly basis. During the summer months of 2023, I supported Nicky in his successful application for a five-week internship funding package to sponsor this continued activity. Since the fall of 2023, Nicky has supported himself as a small-group supervisor of undergraduates and Master's students at Cambridge.

The Master's project which I allocated to Nicky was on the topic of lattice quantum gravity, so-called \emph{dynamical triangulations}. This topic called for familiarity with the path integral formulation of quantum field theory, as well as classical general relativity and differential geometry. Nicky had at the time a poor foundation in differential geometry from his undergraduate lectures, but had independently strengthened this with his own reading to an extent which impressed me at interview. This, and his evident enthusiasm and determination to pursue a career in theory persuaded me to offer him the project, despite apparently lacklustre exam scores throughout his undergraduate degree. I discovered through a combination of reference letters and direct experience, that these low paper scores were purely an artefact of Nicky's complicated neurodiversity and mental health profile. In short, Nicky has severe ADHD, which makes him an indifferent lecture-hall student but an excellent researcher. On top of this, he has crippling, acute performance-related anxiety, which sporadically affects his results in exams and at interview. In a comfortable research setting, this anxiety seems to evaporate. Nicky's exam results as a final-year Master's student were very solid, and representative of his true caliber as a physicist.

Nicky performed exceptionally well during his Master's research project. He was particularly focussed on developing the first \emph{geometric algebra} formulation of dynamical triangulations. Geometric algebra is a unifying mathematical language for physics, combining the concepts of graded, Clifford and Grassmann algebras. Nicky was able to self-teach the fundamentals of geometric algebra during the project. He opened various new research avenues, and successfully completed the research goals. I do not have access to the final score of Nicky's thesis, but I do know it was rated very highly (overall First Class). Nicky and I are putting the finishing touches to a paper which will be submitted to \emph{Advances in Applied Clifford Algebras}. A distinguishing mark of Nicky's research style is his productive tendency to become obsessed with a problem. This is apparently an ADHD-related feature, which Nicky has learned to channel to great effect.

Nicky was not successful with his Ph.D. applications during the 2022-23 cycle. He applied to my group, and I eagerly awarded him a supervisorial offer: however the departmental panel failed to translate this into a funded offer, due to Nicky's low paper-scores up to that date. Nicky is absolutely determined to become a professional theoretician, with a quantum gravity focus. During the current cycle, he is applying accordingly, and to fill the gap he has begun supervising various courses (including general relativity) for undergraduates here at Cambridge. Note that this kind of teaching is something that we typically only allow strong doctoral students to perform; it is clear that the Physics Department (Cavendish Laboratory) has already placed its trust in Nicky's skills.

In summary, Nicky is a strong and determined theoretical physicist. He is a far stronger researcher than he might appear from his exam results, and possibly far stronger than he appears at his Ph.D. interview. I wish his eventual supervisor every good fortune in keeping up with him.
